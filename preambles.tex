% This package allows the use of conditions, making the template more dynamic.
\usepackage{ifthen}

% This is for overwriting the commands.
\newcommand{\declarecommand}[1]{\providecommand{#1}{}\renewcommand{#1}}

% This package is to create dummy texts. You may delete it after there is no 
% dummy text anymore.
\usepackage{blindtext}

% This package is for placing textboxes at the desired positions.
\usepackage[absolute,overlay]{textpos}

% This package for the hyperlinks.
\usepackage[hidelinks]{hyperref}

% This part is for positioning the page numberings.
\usepackage{fancyhdr}
\pagestyle{fancy}
\fancyhf{}
\renewcommand{\headrulewidth}{0pt}
\footskip=0.5cm
\fancyhfoffset[RO,L]{0.5cm}
\rfoot{\thepage}
\makeatletter
\renewcommand\chapter{\if@openright\cleardoublepage\else\clearpage\fi
	\thispagestyle{fancy}
	\global\@topnum\z@
	\@afterindentfalse
	\secdef\@chapter\@schapter}
\makeatother

% This package is to set the paper type and margins.
\usepackage{geometry}
\geometry{
	a4paper,
	left   = 3.5cm,
	top    = 2.5cm,
	right  = 2.5cm,
	bottom = 2.5cm
}

% These packages are to set the font as Times New Roman.
\usepackage{newtxtext}
\usepackage{newtxmath}
\usepackage{titlesec}

% To set the chapter headings
\titleformat{\chapter}
	[display]
	{\vspace{-3cm}\centering\normalfont\fontsize{16}{24}\selectfont\bfseries}
	{CHAPTER \thechapter}
	{0.5cm}{\MakeUppercase}
	[\vspace{-0.2cm}]

% To set the section headings
\titleformat{\section}
	[hang]
	{\normalfont\fontsize{14}{14}\selectfont\bfseries}
	{\thesection.}
	{0.5em}{}

% To set the subsection headings
\titleformat{\subsection}
	[hang]
	{\normalfont\fontsize{14}{14}\selectfont\bfseries}
	{\thesubsection.}
	{0.5em}{}

% Set the default font size as 12
\fontsize{12}{12}

% Left one space before and after the headings
\titlespacing*{\section}{0pt}{\baselineskip}{\baselineskip}
\titlespacing*{\subsection}{0pt}{\baselineskip}{\baselineskip}

% It is for the spacing. The guide remarks the spacing must be 1.5. However, 
% the size fits with the guide document when 1.75 in LaTeX.
\usepackage{setspace}
\setstretch{1.75}

% Set spacings
\newcommand{\OneAndHalfSpacing}{\vspace{0.15cm}}
\newcommand{\DoubleOneAndHalfSpacing}{\vspace{1.30cm}}

% Set indentations
\setlength{\parindent}{1.27cm}
\usepackage{indentfirst}

% Set bibliography
\ifthenelse{\equal{\BibliographyPreferences}{chicago}}{
	\usepackage[authordate,backend=biber]{biblatex-chicago}
}{}
\ifthenelse{\equal{\BibliographyPreferences}{acs}}{
	\usepackage[style=chem-acs,backend=biber]{biblatex}
}{}
\addbibresource{references.bib}
\declarecommand{\cite}[1]{\autocite{#1}}

% Set appendices
\RequirePackage[toc,title]{appendix}
\AtBeginDocument{\renewcommand\appendixname{APPENDIX}}
\AtBeginDocument{\renewcommand{\appendixtocname}{APPENDICES}}
\newenvironment{thesisappendices}{
\addtocontents{toc}{\protect\renewcommand{\protect\cftchappresnum}{\appendixname\space}}
\addtocontents{toc}{\protect\vskip20pt}
\addtocontents{toc}{\protect\setlength{\cftbeforechapskip}{1pt}}
\begin{appendices}
}{
\end{appendices}
}

% Set text block center
\newcommand{\TextBlockCenter}{\dimexpr\paperwidth/2-\textwidth/2}

% Set the table of contents
\setcounter{secnumdepth}{5}
\setcounter{tocdepth}{5}
\usepackage[nottoc]{tocbibind}
\RequirePackage[titles]{tocloft}
\renewcommand{\contentsname}{TABLE OF CONTENTS}
\renewcommand{\listtablename}{LIST OF TABLES}
\renewcommand{\listfigurename}{LIST OF FIGURES}
\renewcommand{\cftpartleader}{\cftdotfill{\cftdotsep}}
\renewcommand{\cftchapleader}{\cftdotfill{\cftdotsep}}
\renewcommand{\cftdotsep}{1}
\renewcommand{\cftchapfont}{\normalfont}
\renewcommand{\cftchappagefont}{} 
\renewcommand{\cftpartpagefont}{}
\renewcommand{\cftsecaftersnum}{.}
\setlength{\cftbeforechapskip}{20pt}
\setlength{\cftbeforetoctitleskip}{0pt}
\setlength{\cftbeforeloftitleskip}{0pt}
\setlength{\cftbeforelottitleskip}{0pt}
\renewcommand\cftchappresnum{\chaptername~}
\renewcommand\cftchapaftersnum{.}
\newlength{\mytoclength}
\settowidth{\mytoclength}{\cftchappresnum\cftchapaftersnum}
\addtolength\cftchapnumwidth{\mytoclength}
\setlength{\cftsecindent}{5.5em}
\setlength{\cftsubsecindent}{6.25em}
\setlength{\cftsubsubsecindent}{7em}
\setlength{\cftparaindent}{7.75em}
\setlength{\cftsubparaindent}{8.5em}

% Set lists
\makeatletter
\newlength{\latexpnumwidth}
\setlength{\latexpnumwidth}{\@pnumwidth}
\renewcommand{\cfttabpresnum}{\tablename~}
\newlength{\mylotlength}
\settowidth{\mylotlength}{\cfttabpresnum\cfttabaftersnum}
\addtolength\cfttabnumwidth{\mylotlength}
\let\origcftdot\cftdot
\addtocontents{lot}{\protect\cftsetpnumwidth{25pt}}
\addtocontents{lot}{\protect\renewcommand{\protect\cftdot}{}}
\addtocontents{lot}{\protect\contentsline{table}{\numberline{\textbf{\underline{\tablename}~}}}{\textbf{\underline{Page}}}{}{}}
\addtocontents{lot}{\protect\renewcommand{\protect\cftdot}{\origcftdot}}
\addtocontents{lot}{\vskip\baselineskip\par}
\renewcommand{\cftfigpresnum}{\figurename~}
\newlength{\myloflength}
\settowidth{\myloflength}{\cftfigpresnum\cftfigaftersnum}
\addtolength\cftfignumwidth{\myloflength}
\let\origcftdot\cftdot
\addtocontents{lof}{\protect\cftsetpnumwidth{25pt}}
\addtocontents{lof}{\protect\renewcommand{\protect\cftdot}{}}
\phantomsection\addtocontents{lof}{
\protect\contentsline{figure}{\numberline{\textbf{\underline{\figurename}~}}}{\textbf{\underline{Page}}}{}{}} 
\addtocontents{lof}{\protect\renewcommand{\protect\cftdot}{\origcftdot}}
\addtocontents{lof}{\vskip\baselineskip\par}
\makeatother

% Put a dot after the figure and table references
\renewcommand{\thetable}{\arabic{table}.}
\renewcommand{\thefigure}{\arabic{figure}.}

% Remove the colon in the captions of the figures and tables
\usepackage{caption}
\captionsetup[table]{labelsep=space}
\captionsetup[figure]{labelsep=space}

\BeforeBeginEnvironment{figure}{\vspace{3cm}}
\AfterEndEnvironment{figure}{\vspace{3cm}}

% Other useful packages
\usepackage{url}
\usepackage{amsmath}
\usepackage{siunitx}
\usepackage{tikz}
\usepackage{mathtools}
\usepackage{pgfplots}
\pgfplotsset{compat=1.7}
